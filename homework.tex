\documentclass[UTF8,zihao=-4]{ctexart}
\usepackage{graphicx}
\usepackage{geometry}
\usepackage{siunitx}
\usepackage{amsmath}
\usepackage{amsfonts}
\usepackage{amssymb}
\usepackage{listings}
\usepackage{forest}
\usepackage{tikz}
\lstset{
	language=C++,
	breaklines,
	tabsize=4,
	basicstyle=\ttfamily \small
}
\geometry{a4paper,centering,scale=0.8}
\title{\heiti 人工智能\quad 第三次作业}
\author{PB17000005\quad \CJKfontspec{AR PL UKai CN} 赵作竑}
\date{\kaishu \today}
\begin{document}
	\maketitle
	\begin{enumerate}
		\item[7.13]
		\begin{enumerate}
			\item[a.]题目可以由标准的逻辑等价进行证明:
			\begin{align*}
				((P_1\wedge \cdots \wedge P_m) \Rightarrow Q) &\equiv (\neg (P_1\wedge \cdots \wedge P_m) \vee Q) \\
				\neg ((P_1\wedge \cdots \wedge P_{m-1}) \wedge P_m) &\equiv (\neg (P_1\wedge \cdots \wedge P_{m-1})\vee(\neg P_m)) \\
				&\cdots \\
				(\neg (P_1\wedge \cdots \wedge P_m)) &\equiv (P_1 \vee \cdots \vee P_m) \\
				((P_1\wedge \cdots \wedge P_m) \Rightarrow Q) &\equiv (P_1 \vee \cdots \vee P_m \vee Q) \\
			\end{align*}
			从而子句$(P_1 \vee \cdots \vee P_m \vee Q)$逻辑等价于蕴含语句$(P_1\wedge \cdots \wedge P_m) \Rightarrow Q$。
			\item[b.]由上一问,我们知道$(P_1\wedge \cdots \wedge P_m)\Rightarrow (Q_1\vee \cdots \vee Q_n)$逻辑等价于
			\begin{equation*}
				P_1\vee \cdots \vee P_m \vee Q_1 \vee \cdots \vee Q_n
			\end{equation*}
			因为命题逻辑的每个语句逻辑都等价于某子句的合取式,而合取式也等价于某种蕴含语句,所以只要把语句转化为合取范式,再从合取范式转换成蕴含范式,就就完成了每个子句到蕴含范式的转换。
			\item[c.]我认为可以分三个步骤:第一步,把所有蕴含范式写成合取范式的形式;第二步,按照合取范式的归结算法进行归结;第三步,将得到的结果转换为蕴含范式。
		\end{enumerate}
		\pagebreak
		\item[证明题]证明Forward Chaining Algorithm的完备性:
		
		每个被蕴涵的原子语句都可以推导得出。验证这一点的最简单的方法是考察\textit{inferred}表的最终状态(在算法到达\textsf{不动点}以后,不会再出现新的推理)。该表把推导出的每个符号设为\textit{true},而其他符号为\textit{false}。可以把此表看作一个逻辑模型;而且,原始\textit{KB}中的每个限定子句在该模型中都为真。为了看清这一点,假定相反情况成立,即某个字句$a_1\wedge \cdots \wedge a_k \Rightarrow b$在此模型下为假。那么$a_1\wedge \cdots \wedge a_k$在模型中必须为真,$b$必须为假。但这和我们的假设即算法已经到达一个不动点相矛盾!因而可以得出结论,在不动点推导出的原子语句集定义了原始\textit{KB}的一个模型。更进一步,被\textit{KB}蕴涵的任一原子语句$q$在它的所模型中为真,尤其是这个模型。因此,每个被蕴涵的语句$q$必定会被算法推导出来。
	\end{enumerate}
\end{document}